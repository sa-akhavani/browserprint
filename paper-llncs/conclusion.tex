\section{Conclusion}
\label{sec:conclusion}

Web browsers have become important tools in our daily lives.
Millions of users use web browsers for a wide range of purposes such
as social media, online shopping, or surfing the web. Some of these
services use browser fingerprinting to track and profile their users
which can be in contrast with their web privacy. At the same time,
browsers are also being targeted by attackers because they are
attractive platforms to compromise. In this paper, we analyzed the
impact of browser features on browser fingerprinting and security. We
investigated more than 30 major browser versions for both Google
Chrome and Mozilla Firefox between 2016 and 2020.

First, we extracted every browser feature that existed in these
browser versions using the browser APIs. Then, we analyzed the feature
sets for these browsers and compared them. One key observation was
that the feature numbers are overall increasing in modern browsers,
and they are indeed becoming more ``bloated'' in general.

Next, we compared the feature reports for these browsers to the
already listed fingerprinting APIs in browsers that are presented in the
literature. Our findings suggested that each browser version between 2016
and 2020 was uniquely fingerprintable, and that the fingerprintablity
of the browsers have been increasing over the years.

We also analyzed if the increase in the numbers of supported features
and the fingerprintability lead to more reported security
vulnerabilities. We extracted the CVEs for every browser version, and
compared the number of CVEs to the number of features supported in the
browser. Our data suggest that there is no direct relationship
between the number of reported vulnerabilities in a browser and the number of
total features that it supports. Contrary to common belief, the
``bloating'' of browsers does not seem to be leading to more
practical exploitability.
