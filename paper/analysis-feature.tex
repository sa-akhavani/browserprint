The first analysis we performed on the dataset we collected was to
understand how browser features have evolved over time. As we describe
in Section~\ref{sec:methodology}, we consider \textit{browser features}
all functionality exposed to JavaScript as objects, methods, and property values.
This definition of browser features reflects on 1) how attackers craft web attacks
(i.e., creating a unique fingerprint using such features, or exploiting  vulnerabilities) and 2)
a measurable metric across browser versions. Understanding
and gaining insights into how browsers are dealing with new
as well as older features is important to be able to distill
conclusions about how secure and fingerprintable browsers are becoming
as they evolve. Hence, our analysis looked at specific browser
features that were introduced, what the typical lifespan of features
looks like.

After extracting feature information for all of the browsers under
analysis, we automatically parsed the generated reports and analyzed
them to see if the features in these browsers fall into specific
categories. Our analysis suggested that the features in Firefox, Opera, and
Chrome can be categorized into three main categories:

\begin{itemize}
  
\item \textbf{Persistent Features}: These are features that are added to a
  specific version, and that continue to exist in every version that
  is released after the feature was introduced. We consider a feature
  to be ``persistent'' if it appears in at least two distinct browser
  versions.
      
\item \textbf{Non-Persistent Features}: These are features that existed in
  older versions of the browser, but were removed, and never appeared
  in newer versions of the browser again. We consider a feature to be
  ``non-persistent'' if it is absent in at least two distinct versions
  of the browser versions under analysis.
      
\item \textbf{Recurring Features}: These are features that are added and
  removed from the browser from time to time. That is, they are
  introduced, they are removed, and they might appear again at some
  point. Such features are typically being tested by the vendors, and
  it is not clear if they will become persistent, or non-persistent.

\end{itemize}

Our analysis suggests that Chrome possesses 9,718 persistent, 711
non-persistent, and 3,161 recurring features that it supports. Similarly,
Opera contains 9,674 persistent, 711 permanently removed, and 3,219 recurring
features. On the other hand, Firefox supports 6,274 persistent, 809 non-persistent, and
115 recurring features. Note that Firefox, overall, supports significantly fewer
features than Chrome and Opera. Also, our analysis suggests that Firefox,
compared to Chrome and Opera, is keeping fewer features (i.e., they are removing
more) over time. Figure \ref{fig:feature-categories} illustrates the feature categories for
each browser vendor. It can be seen that Opera and Chrome are having similar patterns since
lots of their features are related to Chromium which is their shared codebase.
Besides, Chrome and Opera have a greater portion of recurring features compared to Firefox.
This means that Chrome and Opera tend to do more experiments
on adding and removing specific features through time.

\begin{figure}[ht]
  \centering
  \includegraphics[width=.9\columnwidth]{figures/feature-categories-bar.png}
  \caption{Feature category distribution for browsers.}
  \label{fig:feature-categories}
\end{figure}

% \begin{figure}[ht]
%     \centering
%     \includegraphics[width=\columnwidth]{figures/chrome-feature-categories.png}
%     \caption{Feature category distribution for Chrome.}
%     \label{fig:chrome-categories}
% \end{figure}

% \begin{figure}[ht]
%     \centering
%     \includegraphics[width=\columnwidth]{figures/firefox-feature-categories.png}
%     \caption{Feature category distribution for Firefox}
%     \label{fig:firefox-categories}
%   \end{figure}


%   \begin{figure}[ht]
%     \centering
%     \includegraphics[width=\columnwidth]{figures/opera-feature-categories.png}
%     \caption{Feature category distribution for Opera}
%     \label{fig:opera-categories}
%   \end{figure}

  In this work, we also performed an analysis of the common features
  between Firefox, Chrome, and Opera. Since 2016, the total number of features
  introduced by these browsers is 15,945. Among all these features,
  there exist only 4,843 common features among Firefox and Chrome -- which is
  approximately 30\% of the total number of features that these
  vendors support. This number is the same between Firefox and Opera too, with
  4,843 common features between them. On the other hand, Chrome and Opera
  have a bigger set of common features. There exists 13,558 
  common features between Opera and Chrome -- which is approximately 85\% of the total
  number of features that these vendors support. The impact of this huge common features set
  on fingerprintability between two browsers are analyzed in the next section.

  We can conclude that Firefox does not have a high overlap of the features with Chrome and Opera.
  Note that although these browsers often offer very similar functionality,
  unsurprisingly, their codebase might be very different from each other.
  We are aware that Firefox's codebase is very different from Chrome's and Opera's.
  Hence the API names through which these features are available are also often
  significantly different. 
  % As a result, it is clear that vulnerabilities in these browsers will be very specific to the version and vendor. 
  To the contrary, Chrome and Opera share the same codebase.
  This leads to having a bigger set of common features between these two browsers. % Yet, as shown in Figure 1, how these features are enabled and when plays a significant role in distinguishing the two browsers, despite their shared codebase.

  Figures \ref{fig:ffaddremove} and \ref{fig:chaddremove} show the
  feature addition and removal trends for Firefox and Chrome. The data
  shows that Chrome is adding and removing many more features than
  Firefox in each version that is released if one looks at the overall
  numbers of features. However, Firefox seems to be more constant with
  respect to the number of new features added, and older features
  removed. Hence, Firefox seems to be more aggressive with respect to
  removing older features from the browser, ``debloating'' this way 
  the browser. Chrome and Opera share the same trend, so we omit a separate figure for Opera and leave Figure~\ref{fig:chaddremove} as a representative visualization of feature introduction and removal for Chromium-based browsers.

\begin{figure}[ht]
    \centering
    \includegraphics[width=\columnwidth]{figures/Firefox-add-remove.png}
    \caption{Feature introduction and removal in Firefox.}
    \label{fig:ffaddremove}
\end{figure}

\begin{figure}[ht]
    \centering
    \includegraphics[width=\columnwidth]{figures/Chrome-add-remove.png}
    \caption{Feature introduction and removal in Chrome.}
    \label{fig:chaddremove}
\end{figure}

By using the feature datasets we extracted from the Firefox, Opera, and Chrome
versions, we compared feature trends for these browsers. The trends are
depicted in Figure \ref{fig:featuretrends}. The graph shows that the
  number of features supported by Firefox seems to be quite steady
  (i.e., if new features are added, some older ones are typically
  removed) while the number of features supported by Chrome and Opera is growing
  over time. Hence, the data suggests that Chrome and Opera are
  following differing browser feature development philosophies compared to Firefox.

\begin{figure}[ht]
    \centering
    \includegraphics[width=\columnwidth]{figures/Feature-Trends.PNG}
    \caption{Feature trends in Firefox, Opera, and Chrome when compared to
      each other.}
    \label{fig:featuretrends}
\end{figure}
