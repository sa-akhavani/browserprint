\begin{abstract}

Web browsers are indispensable applications in our daily lives.  Millions of
users use web browsers for a wide range of activities such as social media,
online shopping, emails, or surfing the web.  Some of these services use browser
fingerprinting to track and profile their users with clear disregard for
their web privacy. At the same time, browsers are also being targeted by
attackers because they are attractive platforms to compromise.  In this paper,
we perform an empirical analysis of a large number of browser features and aim
to evaluate browser fingerprintability. By analyzing
33 Google Chrome, 31 Mozilla Firefox, and 33 Opera major browser versions released through
2016 to 2020, we discover that all of these browsers have unique feature sets
which makes them different from each other. By comparing these features to the
fingerprinting APIs presented in literature that have appeared in this field, we
conclude that all of these browser versions are uniquely fingerprintable. Our
results show an alarming trend that browsers are becoming more fingerprintable
over time because newer versions contain more fingerprintable APIs compared to older ones.

\end{abstract}
