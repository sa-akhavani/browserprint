\section{Conclusion}
\label{sec:conclusion}

The evolution of the web relies on browsers adding new features that drives innovation in web applications. Yet, this innovation comes at a significant cost to the end users' privacy, since browser fingerprinting techniques abuse certain browser features. In this paper, we analyzed the
impact of browser features on browser fingerprinting. We
investigated more than 30 major browser versions for Google
Chrome, Mozilla Firefox, and Opera between 2016 and 2020.

First, we extracted every browser feature that existed in these
browser versions using the browser APIs. Then, we analyzed the feature
sets for these browsers and compared them. One key observation was
that the feature numbers are overall increasing in modern browsers,
and they are indeed becoming more ``bloated'' in general.

Next, we compared the feature reports for these browsers to the
already listed fingerprinting APIs in browsers that are presented in the
literature. Our findings suggested that each browser version between 2016
and 2020 was uniquely fingerprintable, and that the fingerprintablity
of the browsers has been increasing over the years.

We envision our research to affect how browser vendors introduce new features and take into consideration the effects that these have on browser fingerprintability. Our goal is to highlight the concerning trend of ``bloating'' in the browser and encourage browser vendors to remove abused features in order to improve privacy on the web.

% We also analyzed if the increase in the numbers of supported features
% and the fingerprintability lead to more reported security
% vulnerabilities. We extracted the CVEs for every browser version, and
% compared the number of CVEs to the number of features supported in the
% browser. Our data suggest that there is no direct relationship
% between the number of reported vulnerabilities in a browser and the number of
% total features that it supports. Contrary to common belief, the
% ``bloating'' of browsers does not seem to be leading to more
% practical exploitability.
