\section{Conclusion}
\label{sec:conclusion}

In this paper, we analyzed different impacts of browser features in browser fingerprinting and security. We investigated more than 30 major browser versions for both Google Chrome and Mozilla Firefox. First, we extracted every browser feature that existed in these browser versions using the browser APIs. Then, we analyzed the feature set for these browsers and compared them together. We found out that the feature number is increasing in newer versions.
Next, we compared feature reports for these browsers to the already listed fingerprinting APIs in browsers that are presented in different papers of the field. We saw that all of these browsers have some of these fingerprinting APIs in them so we say that all of the browsers are fingerprintable. Also, newer browsers are including more of these fingerprintg APIs thus they are more fingerprintable than the older versions. 
Finally, we tried to find out whether this increase in the browser features is leading to more security flaws or not. We extracted CVEs related to these browser versions and compared the number of CVEs for each browser version to the number of features in it. Our data did not show that there is a valid relation between these two. So we can say that browsers are becoming more secure and newer browsers are having fewer vulnerabilities.