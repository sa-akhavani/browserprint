\section{Research Questions}
\label{sec:research}

In this paper, by performing an automated analysis, we attempt to
answer the following research questions:

\begin{enumerate}
  
\item {\em Are major versions of Firefox, Chrome, and Opera browsers
  fingerprintable?} Our results suggest that the feature set for each
  browser version is unique. There exist multiple
  APIs in every browser version that we have analyzed that can be used
  for fingerprinting. By extracting all the features supported by a
  browser and exposed via API calls, we can uniquely identify each
  browser version.

\item {\em Are Firefox, Chrome, and Opera becoming more
  fingerprintable over time?} One of the major conclusions of our
  study is that the number of APIs one can use in the newer versions
  of Chrome, Opera, and Firefox is larger than the older versions. Hence, newer
  browser versions are even more fingerprintable than previous
  versions, and our findings suggest that this trend is likely to
  continue. As a result, privacy might be an even more significant
  concern in the future for browser users.
    
\item {\em What ``lifespan profiles'' can we cluster browser features
    into? Are there any``permanently removed'' features? If so, how
    does their life cycle look like?} Our results suggest that we can
  categorize browser features based on their lifespan into three main
  categories (i.e., persistent features, non persistent features, and
  recurring features). We observe that most of the features are added
  permanently, and are not removed over time -- indicating that
  browsers are indeed becoming more ``bloated'' as they evolve.

\item {\em With respect to browser bloating, how does Firefox compare
    to Chrome and Opera?} In our study, we were able to map the number of unique
  features for major versions of Firefox, Chrome, and Opera. The results
  suggest that Chrome and Opera are introducing more features over time than
  Firefox, but that all of these browser vendors have shown a significant
  increase in the total number of features they support per version
  since 2016. Compared to Firefox, Chrome and Opera tend to introduce more new
  features and keep them around longer.
 
% \item {\em Is there a correlation between the number of features
%     available in a browser (i.e., how ``bloated'' the browser is) and
%     the number of vulnerabilities that exist on that browser?} Our
%   data suggest that, unlike the widely held belief, there is no
%   direct correlation between the number of features that a browser
%   version supports and the number of vulnerabilities in that
%   version. Although browsers are indeed becoming more ``bloated'' over
%   time, at the same time, their codebase seems to be becoming more
%   secure.

\item {\em Could the incognito mode in Chrome and the private window mode in
  Firefox and Opera reduce the possibility of being fingerprinted by websites?} Our analysis
  suggests that the incognito and private window modes have negligible impact on
  reducing fingerprinting. That is, almost all fingerprinting APIs are accessible
  in these modes the same way that they are available in non-private mode.
    
\item {\em Although Opera and Chrome are both Chromium-based and share the same codebase,
  is there any noticeable difference between these two browsers in case of fingerprintability?}
  In our analysis, we find out that Opera and Chrome have very similar sets of fingerprintable APIs
  and there is not much difference between these two browsers in case of fingerprintability. But there exist differences in some browser-specific features between these two browsers.
  Additionally, Opera and Chrome follow almost the same pattern in feature adding and removal as a result of their shared codebase. These browsers tend to keep a majority of their features untouched.
\end{enumerate}

