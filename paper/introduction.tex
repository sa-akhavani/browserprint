\section{Introduction}
\label{sec:introduction}

Web browsers have become indispensable in our daily lives. The
majority of the online activity of many Internet users comprises of
using a browser to access social media, online shopping, surfing the
web, messaging, and accessing stored information in the cloud.
Unfortunately, as browsers have become critical and important
infrastructure components in our lives, they have also become
attractive targets for attackers who are looking into compromising
them. At the same time, many companies are interested in collecting
the private browser activities of end-users for marketing and sales
purposes. To achieve their data collection objectives, some web
services use ``browser fingerprinting'' to track and profile their
users. Clearly, browser fingerprinting can be in stark contrast with
the web privacy of browser users.

As browsers are becoming increasingly entwined and integrated into the
operating systems, many unique details of a user's browser such as its
timezone, operating system, configuration and preferences can be
exposed through the browser. An attacker who collects and sums these
outputs can create a unique ``fingerprint'' for tracking and
identification purposes. In addition, browsers have also been
increasing in complexity as more and more new features are being
integrated into them, raising concerns that the attack surface offered
by this software ``bloating'' (i.e., the increase in the number of
components and code not needed by every user) is contributing to
making browsers more difficult to secure against attacks.

Browser fingerprinting has been determined to be an important problem
by previous research (e.g.,~\cite{}) as well as browser vendors
themselves (e.g.,~\cite{}). To date, however, no studies have looked
at popular browsers historically and have attempted to determine how
their fingerprintability has evolved over the years. Analogously, we
are also not aware of studies that have looked into how, and if, the
increase in the number of new features integrated into the browsers
every year have contributed to an increase in the number of reported
vulnerabilities.

In this paper, we perform an empirical analysis of a large number of
browser features that have been integrated or phased out of the
popular Mozilla Firefox and the Google Chrome browsers between the
years 2016 and 2020. Our aim is to answer a number of research
questions about the fingerprintability and security of these browsers
over this time period. By analyzing 33 Google Chrome and 31 Mozilla
Firefox major browser versions, our results show that these popular
browsers have unique feature sets that make them significantly
different from each other. Hence, by comparing these features to the
fingerprinting APIs presented in literature, we conclude that all of
these browser versions are uniquely fingerprintable. Our results show
the alarming trend that browsers are becoming more fingerprintable
over time as newer versions of popular browsers have more
fingerprintable APIs embedded in them. Another key finding of our
study is that unlike the popular folk wisdom that software bloating
always leads to more security vulnerabilities~\cite{Bloating}, the
number of vulnerabilities in browsers are not directly related to the
number of new features being supported by those browsers. In fact, our
measurements show that browsers are becoming more secure over time
even though they are supporting more features and are becoming more
``bloated''.

In this paper, by performing multiple analysis on the data that we
have collected, we attempt to answer the following questions:

\begin{enumerate}
  
\item Can we say that all browsers are fingerprintable? Yes. We show
  that the feature set for each browser version in Chrome
  and Firefox is unique. Also, there exist multiple fingerprinting
  APIs in every browser version we have analyzed. So we can identify
  each browser version by extracting its features and all browsers are
  fingerprintable.
  
  \item What "lifespan profiles" can we cluster browser features into?
    Are there any "permanently removed" features, and if so, how
    old/established were they? How long is the feature lifespan of
    removed features? We can categorize features based on their
    lifespan into 3 different categories. We see that most of the
    features are permanently added but there exists many permanently
    removed features in both Chrome and Firefox. We discuss this in
    details in section 2.1.
    
  \item With respect to bloating, if we map the number of features per
    browser, how does the trend look? The number of features per
    browser is increasing. It is more intense in Chrome and they are
    adding much more features than Firefox but both of the browser
    vendors have had a significant increase in the number of features
    per browser in the 4 year period that we analyzed.
    
  \item Who is leading and who is following in feature introduction
    among browser vendors? Are we seeing some major differences in
    terms of feature introduction and removal between Firefox and
    Chrome? The feature introduction and removal are not similar
    between Chrome and Firefox. Firefox tends to keep its feature
    numbers steady. They remove a bigger portion of their features
    compared to Chrome. But chrome tends to add more features and
    keeps them in the browser. So there is not a similarity between
    Chrome and Firefox in terms of feature introduction.
    
  \item Is there a correlation between the number of features in a
    browser and the number of vulnerabilities that exist on that
    browser? Based on our data and analysis, there is not a valid
    correlation between the number of browser features and the number
    of vulnerabilities. Although we are having many more features in
    the newer browsers, they are becoming more secure and the number
    of features does not affect the security of browsers.
    
  \item Are browsers becoming more fingerprintable? Yes. One major
    finding of our analysis is that number of fingerprinting APIs in
    newer browsers is bigger than the older versions so we can say
    that newer browsers are becoming more fingerprintable. versions.
    
\end{enumerate}

The rest of the paper is organized as follows: The next section
describes our methodology and data gathering technique. Section 3
presents a different analysis based on the feature reports and their
relation to browser vulnerability. In Section 4, we present the
related work and then briefly conclude the paper in Section 5.