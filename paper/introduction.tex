\section{Introduction}
\label{sec:introduction}

Web browsers have become indispensable in our daily lives. The
majority of the online activity of many Internet users comprises of
using a browser to access social media, online shopping, surfing the
web, messaging, and accessing stored information in the cloud.
Unfortunately, as browsers have become critical and important
infrastructure components in our lives, they have also become
attractive targets for attackers who are looking into compromising
them. At the same time, many companies are interested in collecting
the private browser activities of end-users for marketing and sales
purposes. To achieve their data collection objectives, some web
services use ``browser fingerprinting'' to track and profile their
users with clear disregard for their web privacy.

As browsers increasing supplant traditional operating systems as the
application publishing platforms of choice, 
many unique details of a user's browser such as its hardware,
operating system, browser configuration and preferences can be exposed
through the browser. An attacker who collects and sums these outputs
can create a unique ``fingerprint'' for tracking and identification
purposes. In addition, browsers have also been increasing in
complexity as more and more new features are being integrated into
them, raising concerns that the attack surface offered by this
software ``bloating'' (i.e., the increase in the number of components
and code not needed by every user) is contributing to making browsers
more difficult to secure against attacks.

Browser fingerprinting has been determined to be an important problem
by previous research
(e.g.,~\cite{cookiemonster-SP13,panopticlick,mowery2012pixel,fpdetective})
as well as browser vendors themselves (e.g.,~\cite{safari-privacy,brave-fpbudget,firefox-fingerprinting}).
 To date, however, no
studies have looked at popular browsers historically and have
attempted to determine how their fingerprintability has evolved over
the years.
Past work has demonstrated that the ability to simply fingerprinting a browser's precise version without relying on possibly spoofed User-Agent strings can be useful to attackers~\cite{schwarz2019javascript}.
In the further light of web privacy research showing the potential and/or real-world exploitation of novel APIs for fingerprinting \cite{olejnik2017, acar2014web, englehardt2016online}, we consider the raw volume of implemented APIs to be a rough but useful proxy estimate of a browser's potential fingerprintability.


% Analogously, we are also not aware of studies that have
% looked into how, and if, the increase in the number of new features
% integrated into the browsers every year have contributed to an
% increase in the number of reported vulnerabilities.

In this paper, we perform an empirical analysis of a large number of
browser features that have been integrated or phased out of the
popular Mozilla Firefox, the Google Chrome, and the Opera browsers between the
years 2016 and 2020. We consider browser features to be all
functionality that is available to attackers directly through
JavaScript, since these are the root problem of most web attacks. Our
aim is to answer a number of research questions about the
\textit{fingerprintability} and security of these browsers over this
time period. We propose a new metric for quantifying the
fingerprintability of browser versions that rely on the number of
browser features that are associated with fingerprinting. This metric is based on
previous research and current fingerprinting techniques discovered in
the wild (see Section~\ref{sec:fp-apis} for more details). By
analyzing 33 Google Chrome, 31 Mozilla Firefox, and 33 Opera major browser
versions, our results suggest that these popular browsers have unique
feature sets that make them significantly different from each
other. Hence, by comparing these features to the fingerprinting APIs
presented in literature, we conclude that all of these browser
versions are uniquely fingerprintable. Our results suggest the
alarming trend that browsers are becoming more fingerprintable over
time as newer versions of popular browsers have more fingerprintable
APIs embedded in them. 
\ali{SHOULD WE REMOVE THIS? Another key finding of our study is that unlike
the popular folk wisdom that software bloating always leads to more
security vulnerabilities~\cite{Bloating}, the number of
vulnerabilities in browsers are not directly correlated to the number
of new features being supported by those browsers.}
% In fact, our
% measurements suggest that browsers are becoming more secure over time
% even though they are supporting more features and are becoming more
% ``bloated''.

This paper makes the following key contributions:

\begin{itemize}

\item We show that all major Mozilla Firefox, Google Chrome, and Opera browser
  versions between 2016 until 2020 are uniquely fingerprintable.

\item We analyze Mozilla Firefox, Google Chrome, and Opera and report
  major differences between feature introduction and removal
  trends. While Firefox tends to keep a steady number of features in
  the browser (i.e., introducing new features while removing older
  ones), Chrome, in contrast, is growing and more features are kept as
  the browser evolves. Opera, similar to Chrome, seems to be adding lots
  of features and not interested in removing the older ones.

\item We show that although Google Chrome and Opera are both based upon
  Chromium and share the same codebase, there are still differences in their
  feature introduction and removal patterns. But this shared codebase makes them
  very similar in our fingerprintability analysis.

% \item As Firefox and Chrome are becoming more ``bloated'' over time
%   and as their sizes increase, there is no direct correlation between
%   the total number of features in the browsers and the number of
%   vulnerabilities that are reported.

\item We provide all the source code and datasets that we have collected in our experiments to the community.
\footnote{\url{https://anonymous.4open.science/r/52ebbe26-5827-464a-b6e7-bc9d8c1101bb}}

\end{itemize}

The rest of the paper is organized as follows: The next section lists
the research questions we aimed to answer in this study. Section
\ref{sec:methodology} describes our methodology and data gathering
techniques.
% Section \ref{sec:analysis} presents a different analysis
% based on the feature reports and their relation to browser
% vulnerability.
In Section \ref{sec:related-work}, we present the
related work and then briefly conclude the paper in Section
\ref{sec:conclusion}.
